\documentclass[a4paper]{llncs}
\usepackage[utf8]{inputenc}
\usepackage{times,verbatim} % Please do not comment this

\begin{document}
	
	\pagestyle{empty}
	
	\mainmatter
	
	\title{Cybersecurity in IT-Scientist Network Infrastructures by
		Honeypots: Catching Cyber Threat Passively}
	% El título no se entiende bien. Voy a por un issue. - JJ
	\titlerunning{Cybersecurity in IT-Scientific Network Infrastructures by Honeypots}
	
	\author{Juan Luis Martin Acal \and Pedro A. Castillo Valdivieso
		\and Gustavo Romero López \and Juan Julián Merelo Guervós}
	
	\authorrunning{Juan Luis Martin Acal}
	
	\institute{Springer-Verlag, Computer Science Editorial III,
		Postfach 10 52 80,\\
		69042 Heidelberg, Germany\\
		\email{jlmacal@correo.ugr.es}\\
		\email{\{pcv, gustavo\}@ugr.es}\\
		\email{jmerelo@ugr.es}\\
		\texttt{http://www.springer.de/comp/lncs/index.html}
	}
	
	\maketitle
	
	\begin{abstract}
		When dealing with security concerns in the use of IT infrastructures a good balance between security concerns and the right to privacy should be maintained. This is very important in scientific networks, because they were created with an open and decentralized philosophy, in favour of the transmission of knowledge, when security was not a essential topic.
		
		%\cite{iris-proyecto}.  
		%No se ponen citas en el abstract - JJ
		% Deberías hablar de las universidades en el título. NO IT-scientist - JJ
		%FERGU: Bueno, y aunque no está mal del todo, en papers en inglés deberías citar papers en inglés. Busca una referencia equivalente a la que usas, pero en inglés, que queda más fino.
		%JL ME GUSTA BUSCARE REFERENCIAS EN INGLES PARA EL TEMA DE LA FINURA.
		Although private and scientific information have a enormous value for an attacker, the user privacy for legal and ethical reasons must be respected. For this reason passive detection methods in cybersecurity such as
		honeypots are a good strategy to achieve this balance between security and privacy in the defence plan of a scientific network. In this paper we present the practical case of the University of Granada in the application of
		honeypots for the detection and study of intrusions which avoid
		intrusive techniques such as the direct analysis of the traffic through
		networking devices. 
		
		%Casos de exito - JL
		%https://www.honeynet.org/about,
		%https://es.scribd.com/doc/174841947/Memoria
		
		%Para el futureworks -JL
		%Also we expose the difficulties found in the information's analysis gathered. ESTO SER DESCRIBE EN SU SECCION PERO NO SE PRESENTA.
		%Finally we propose the application of machine learning to overcome them in future works. ESTO SERA LA PROPUESTA PARA FUTURE WORKS
		% Future work no va en el abstract. SE cuenta qué se va a hacer - JJ
		
		%FERGU: 
		%Tampoco se suele poner el uso del future work en los abstract (si usas esos métodos de ML en el paper quita lo de future works). Deja claro en el abstract cual es el objetivo de este trabajo
	\end{abstract}
	
	
	\section{Introduction}
	From the earliest days of network communications, these have been experiencing continuously increasing number of attacks\cite{esset-tendencias, cni-ccn-tendencias-2014}. Also, the complexity of these attacks against the information and resources  in the networks has increased. This increase is motivated by  economic, politic or military interests or by the same entities interested in exercise a bigger control over communication freedom in The Internet\cite{cni-ccn-tendencias-2014, cni-ccn-tendencias-2015}.
	%Informe CISCO NSA
	
	Nowadays, although private and scientific information have a enormous value for an attacker, the user privacy for legal and ethical reasons must be respected.
	%FERGU: With the upswing of Internet, the number of attacks on networks have been increased significantly.
	%JUANLU: Desde el primer dia. Tema a parte es que no se documentara ni existan fuentes oficiales porque las empresas y los gobiernos lo ocultaban.
	Scientific networks are a special and interesting case, in one hand there is a strong demand of security in the network and the resources and services which are listening. On the other hand, the end users demand privacy in his network traffic covered by the law. But it was not designed thinking in security concerns\cite{iris-proyecto}.
	%\subsection{Cyber-Space and Cyber-Threats}
	%\label{sect:Scientist Networks}
	%The cyberspace is a virtual space that wraps all types of digital communication  infrastructures and the entities that use them. The hostile actions from these entities against the security and safety of the information and others entities are ciberthreats. Internet is the most popular inhabitant of this space and for years we have seen how
	
	In contrast to the corporate networks which usually have grown from
	the inside to outside and which have most hosts behind the demilitarized
	zone (DMZ), the scientist networks were born with a open philosophy without focusing on security but on technical requirements due to the limited number of public IPs, were expanding private services to the intranet \cite{iris-proyecto}. The information related to research, patents, computer and human resources is a juicy target for hostile agents. Also, the big size of the DMZ makes it prone to a massive attack and increases the possibility of finding a security breach or hidden advance vectors of attack. In this scene the honeypots have an important role in the detection and protection against cyberattacks.
	
	%EN ESTE ARTICULO EXPONEMOS COMO SE COMPONE  SE ANALIZA BLABLABLABLA 
	
	\section{Deployment of a Security System Based in Honeypots}
	\subsection{Honeypots}
	A honeypot is a trap that exposes itself, while is scanned, probed or compromised by a hostile entity, the trap collect information about the malicious activity. We differentiate between hierarchy and interaction in our taxonomy. The hierarchy is the complexity goes from a simple service like SSH, through a network, to a cloud. The interaction is the degree of fidelity in the response of the trap and goes from low to high.\\
	%FERGU: para enlazarlo con lo anterior puedes empezar con "To address all this issues, the honeypots have been used as method to... "  y con citas.
	%Mencionas a la taxonomia sin presentarla. Pon "In this paper we propose a taxonomy to categorize lo que sea".
	%"The hierarchy is the complexity goes from" No sé qué quiere decir esto, reescribelo mejor.
	TABLE HERE.
	
	There isn't a ideal configuration of features because 
	%FERGU: No usar contracciones: isn't is not. a ideal-> an ideal.
	is the nature of the threats and the infrastructure which we want to protect, the key for a correct selection of them.
	%FERGU: mejor reescribirlo como "There is not an ideal configuration of correct features for LOQUESEA, as there exist a huge effect on the nature of the treats and the infrastructure to LOQUESEA."
	In a software development environment, high interactions is used for 
	%FERGU: is used->are used.
	test a new product with {\it fuzzers} or another type of pentesting\footnote{Penetration Testing.}
	%FERGU: Cita describiendo fuzzers. Mejor poner "or another type of Penetration Testing (also called {\it pentesting}) tool \cite{algo}". Pon citas a todos los conceptos que uses
	tool in order to discover potential vulnerabilities. On the other hand, low interaction honeypots are used like intrusion detection systems, warning about activity of scans or jumping attempts from compromised internal hosts. Both share a common point: they are not intrusive with the network traffic.
	
	%FERGU: Aqui irá el clásico párrafo de fin de introducción: En este paper presentamos tal. La sección \ref{sec:soa} describe el estado del arte en TAL. A continuación, en la sección \ref{laquesea} describimos tal ... Finalmente, se presentan las conclusiones y trabajo futuro.
	
	The architecture of the system is divided in two fronts: detection and management of the ciberthreats. The detection front usually are based on honeypots, one the most valuable tools at hand for this purpose.\\
	%FERGU: no uses los \\ si no son necesarios, LaTeX se encarga de ponerlo bonico todo. También empieza la sección diciendo "Teniendo en cuenta lo que hemos expuesto anteriormente en el estado del arte, existen una serie de cosas que nosotros aprovechamos, el tal y el tal. En esta sección describimos..." Todas las secciones deben presentarse (esto es muy importante para el estilo del artículo).
	Sensors were deployed in different production subnets and each content honeypot software. Specifically Dionaea\cite{dionaea} and Kippo\cite{kippo} which are low and medium interaction honeypot respectively. 
	%FERGU: decir que es media y baja interacción (y citar a donde se describa). Di también: hemos usado baja y media interacción porque nos permite hacer tal...
	Each sensor has local data bases for save the information
	%FERGU: for save->to save. TRUCAZO: memoriza TO DO y FOR DOING. To <infinitivo> y For <gerundio>
	attacks efficiently in space 
	%FERGU: in space?
	while is waiting 
	%FERGU: while it is waiting (repite los pronombres si están muy lejos)
	for its saved 
	%FERGU: its savings? que son sus "saved"?
	in the collector in order to keep the information by duplicate and not to increase the network traffic in case of massive scans or attacks. Obviously, in the time space 
	%FERGU: quita el obviously, los lectores a lo mejor no lo tienen tan obvio (yo no lo sabía xD)
	between information dumps each sensor sends by telegram incidents
	%FERGU: by telegrams? Muy del siglo 19, no? xDDD Explica que es un telegrama (supongo que algo de seguridad) o pon "messages" mejor
	defined by the security operator like critical. The collector is a corporate database that feeds the incidents management system 
	and is 
	%FERGU: and it is (lo que te he dicho de los pronombres en inglés)
	the core of all information analysis.
	
	PONER GRAFICO
	
	\section{Data analyse}
	%FERGU: Data analysis
	For three years each sensor collected information of more of half million of connections. The information's analysis shows the next points:
	%FERGU: no usar contracciones ('s). Pon mejor: The analysis of this high amount of information has provided the next facts/conclusions/sinonimo:
	\begin{itemize}
		\item External attacks are more frequent than internal attacks.
		\item In one hand the most frequent type of external attacks was weak credentials disclosure. On the other hand the most frequent type of internal attacks was malware propagation.
		\item Countries outside OTAN are the most active in the process of scanning and searching for vulnerabilities but curiously most of the intrusions come from OTAN member or member candidate countries. It is important to note that this data is dependent on the geolocation of where they are taken.
	\end{itemize}
	
	The increment of attacks in this decade is a proven fact and the data cached shows us how externals attacks are the most frequent kind of attacks. This matches with studies of big security IT enterprises\cite{verizon-2015}. When we studied in detail the information, we saw many attempts of connection, some from scans to the network infrastructure and others looking for exploit vulnerabilities or services without strong credentials. About the latter ones is necessary to emphasize those that showed a more advanced level in the process of intrusion because were linked to Advance persistent threats (APT). One of the greatest dangers for IT infrastructures of governments, public administrations and companies are advance persistence threats. A cyberthreat is persistent if it is continuous in time and establishes monitoring and control mechanisms with hostile agent, and it is advanced because uses mechanisms in order to hide its activity in the system. Usually APT are related with cyberspying and elite groups of cybercrime and they are attacks directed against a specific infrastructure. For this reason it is a priority to detect and study them.
	
	Malware propagation usually belong to advanced and persistent threats and come accompanied by a multivector attack. For example, the malware infect through the ms08\_67 and immediately attempting connect with a command ad control server (C\&C). In this point, start to protect itself through mutations of his own code, in order to hide from antivirus software, communicate through common protocols like HTTP to avoid rules for outgoing traffic firewall, it use encrypted communication by SSL to prevent the interception of information or generate thousands names of random domain to prevent the take down of the botnet.
	
	But not always is easy follow the clue for rebuilding of a multivector attack. Usually the exploitation of SSH or MySQL weak credentials is the first step to gain the control or access to data in a server, but only a very reduced part shows a clever behavior behind the attack. Between hundred of thousand of connections only a few ones shows access to information to a service, then it use this informatión against other services and finally jumping to other hosts. A bit more frequent is the attempt to privilege elevation but the common behavior is use the base vulnerability in order to use his network and computational resources as soon as possible, in tasks like miner Litecoin\cite{litecoin}, increase the number of nodes for other scans of networks, for a future deny of service attack or use the compromised host like a anonymous proxy.
	
	When we rebuild the trace of the attack, the first advanced behavior that we find is the use of different hosts for scans and attacks and others for the intrusions and the explotations. The attack starts to scan subnets usually from countries without collaborations accord, in our case China. Then of the detection of the vulnerabilities, the exploitation is from Europe or United States. Obviously is not possible to be totally sure about the origin of the intrusion because we dodn't detect the use of proxies, but there isn't sense the use from countries like Chine for hidden its source in the intrusion stage but not in the scan stage. It has more relevance the inverse theory, the attack come from countries in a increase technology stage and many resources, where the culture of security isn't still high, finally the intrusion attempt is from a nearer geolocation of the attacker.
	
	
	\section{Strengths and Weaknesses of Honeypots}
	The strengths of honeypot are were:
	\begin{itemize}
		\item It was not intrusive with network traffic, remained the privacy of infrastructure users. This is a important point because any try of to catch indirect traffic of network would been seen as a threat by other users and a infringement of the use conditions of the network and legality.
		\item The computational and economic resources needed for passive detection are lower because we have only the traffic belong to a potential cyber threat. This an alternative approach to other solutions more expensive like intrusion detection systems based in hardware.
		\item Cyberthreats like advanced malware uses ciphered communications in order to dodge detection systems in the network layer, the only way to catch information its from inside of the compromised node. This is essential if we want analyse how persistence cibertheraths monitors the compromised host and what informations sent outside to its C\&C network.
	\end{itemize}
	
	The weakness of the honeypot are:
	\begin{itemize}
		\item There are many cyberthreats focused in the network layer, usually related with deny of services and spoofing. This information is very valuable because this kind of attack are a very important element not only in simple vector attacks, in multivector advanced and persistent attacks too. Honeypots only fetch information from the application layer so they lose essential information for reconstruction of complex attacks.
		\item The use of passive sensors in a security system must be planned with some extra considerations respect the use of active methods of detection. Those considerations cover strategies of deception and hiding of the sensors and politics of migrations in the infrastructure for avoiding it's location.
		\begin{itemize}
			\item Like others deception tools, honeypots must show itself like interesting target for a attacker and avoid them to be easily recognizable by fingerprint techniques. Default installations and configurations in low an medium interaction honeypot are easily detected by a human attacker or a intelligent threat like advanced malware.
			\item When attacker has knowledge of the infrastructure, honeypots are easily dodged so must be deployed together with politics of use like change its subnets or IP every so often.
			\item High interaction honeypots are dangerous in production environments because the monitored sensor is completely real and all his potential in order to attack its periphery. Usually they are deploying in isolated subnets with outgoing traffic strongly restricted in company of others honeypots, that configuration is called honeynet.
		\end{itemize}
	\end{itemize}
	
\section{Improving Honeypots, Hybrid Machine Learning S.I.E.M}
VECTORES HONYPOT VS CONJUTOS DE DATOS DEL LEARNING MACHINE GROUP
FALTA CHICA, CAPA DE RED, TCPDUMP O IDS PILLANDO DATA

\section{Conclusions and Future Works}
HAY QUE METER MACHINE LEARNING Y MINERIA DE DATOS DE LA BUENA A ESTOS DISPOSITIVOS.

\bibliographystyle{splncs03}
\bibliography{articuloHoneypot}

%https://www.schneier.com/blog/archives/2008/06/it\_attacks\_insi.html
%http://www.verizonenterprise.com/resources/reports/rp\_data-breach-investigations-report-2013\_en\_xg.pdf
\end{document}

