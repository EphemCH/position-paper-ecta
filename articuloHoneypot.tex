\documentclass[a4paper]{llncs}

\usepackage[utf8]{inputenc}

\usepackage{times,verbatim} % Please do not comment this
\input{psfig.sty}

\begin{document}

\pagestyle{empty}

\mainmatter

\title{Cybersecurity in TIC-Scientist Network Infrastructures\\by
  Honeypots:\\Catching Cyber Threat Passively}
% No deberías forzar saltos de página, Latex es muy listo para estas
% cosas - jj

\titlerunning{Cybersecurity in TIC-Scientist Network Infrastructures by Honeypots}
% TIC en inglés es IT - JJ

\author{Juan Luis Martin Acal\inst{1}
\and Pedro A. Castillo Valdivieso\inst{1}
\and Gustavo Romero López\inst{1}} %and JJ? - JJ

\authorrunning{Juan Luis Martin Acal}

\institute{Springer-Verlag, Computer Science Editorial III,
Postfach 10 52 80,\\
69042 Heidelberg, Germany\\
\email{jlmacal@correo.ugr.es}\\
\email{\{pcv, gustavo\}@ugr.es}\\
\texttt{http://www.springer.de/comp/lncs/index.html}
}

\maketitle

\begin{abstract}
There is a balance between security concerns and the right to
privacy. % dónde? Public policy? program design? IT infrastructure
         % design? Lo hay o lo debería de haber?- JJ
  Universities have a high risk of attack as a source of valuable
  information. % ¿Sólo las universidades? ¿Por qué las universidades?
               % - JJ
Private and scientific information have a enormous value for an
attacker but the end user is worried about his privacy too. %Par de
                                %fallos gordos de inglés. Por favor,
                                %tened cuidado- JJ
 For this reason passive detection methods in cybersecurity like
 honeypots are the cornerstone in the defence plan. We expose the
 practical case of the University of Granada in the application of
 honeypots for the detection and study of intrusions. 
% ¿Y cuál es el objetivo? ¿Caso de éxito? ¿Caso de fracaso? ¿Hay algo
% original en la implementación de los honeypots? - JJ
\end{abstract}


\section{Introduction}

From the earliest days, % ¿de qué? -JJ
  networks have been experiencing an increasing number of
  attacks. Nowadays, % pasas de earliest days a nowadays sin solución
                     % de continuidad... algo tendrá que haber en
                     % medio - JJ
 the number of attacks increases continuously and scientist networks
 are a special and interesting case. % cita - JJ
 There is a strong demand of security in the network and the services which are listening. On the other hand, the end users demand privacy in his network traffic. In this scene the honeypots have an important role in the detection and protection against cyber attacks.
% ¿Por qué la universidad? ¿Qué le hace especial? Si tienes una
% universidad y quieres aplicarlo, ¿qué tienes que hacer? - JJ

% No suele haber subsecciones en la introducción. 
\subsection{Cyber-Space and Cyber-Threats}
\label{sect:Scientist Networks}

The cyberspace is a virtual space that wraps all types of digital
%Agh, ciberespacio... ¿todo esto para qué ? - JJ
communication  infrastructures and the entities that use them. The
hostile actions from these entities against the security and safety of
the information and others entities are ciberthreats. Internet is the
most popular inhabitant of this space and for years we have seen how
%agh, internet como habitante del ciberespacio... qué poco me gustan
%los clichés - JJ
the number and complexity of attacks against information and resources
has increased. This increase is motivated by for economic, politic or
military interests or by the same entities interested in exercise a
bigger control over communication freedom in the cyberspace.  % Citas,
                                % citas y más citas - JJ


\subsection{Scientist Networks}
\label{sect:Scientist Networks}
In contrast to the corporate networks which usually have grown from
the inside to outside % inside to outside what? - JJ
and which have most hosts behind the Demilitarized
 Zone (DMZ), %citaaaaaa!!!- JJ
the scientist networks were born with a open philosofy
 without focusing on security but on technical requirements due to
the limited number of public IPs, were expanding private services to
the intranet.

The information related to research, patents, computer and human resources is a juicy target for hostile agents. Also, the big size of the DMZ %citaaaaa! - JJ
 makes it prone to a massive attack and increases the possibility of finding a security breach or hidden advance vectors of attack.

\subsection{Privacy and Passive Sensors}
\label{sect:Privacy and Passive Sensors}
A honeypot is a trap that exposes itself, while is scanned, probed or compromised by a hostile entity, the trap collect information about the malicious activity.

We differentiate between hierarchy and interaction in our taxonomy. The hierarchy is the complexity goes from a simple service like ssh, through a network, to a cloud. The interaction is the degree of fidelity in the response of the trap and goes from low to high.

TABLE HERE.

There isn't a ideal configuration of features because is the nature of the threats and the infrastructure which we want to protect, the key for a correct selection of them. In a software development environment, high interactions is used for test a new product with {\it fuzzers} or another type of pentesting\footnote{Penetration Testing.} tool in order to discover potential vulnerabilities. On the other hand, low interaction honeypots are used like intrusion detection systems, warning about activity of scans or jumping attempts from compromised internal hosts. Both share a common point: they are not intrusive with the network traffic.

\section{Deployment of a Security System Based in Honeypots}

The architecture of the system is divided in two fronts: detection and management of the ciberthreats. The detection front usually are based on honeypots, one the most valuable tools at hand for this purpose.

\subsection{Sensors and Collector}

Sensors were deployed in different production subnets and each content honeypot software. Specifically Dionaea\cite{dionaea} and Kippo\cite{kippo} which are low and medium interaction honeypot respectively. Each sensor has local data bases for save the information attacks efficiently in space while is waiting for its saved in the collector in order to keep the information by duplicate and not to increase the network traffic in case of massive scans or attacks. Obviously, in the time space between information dumps each sensor sends by telegram incidents defined by the security operator like critical.

The collector is a corporate database that feeds the incidents management system and is the core of all information analysis.


\subsection{Attacks Profiles}

For three years each sensor collects information of more of half million of connections. The information analysis shows that prevail external attacks for all types of the vulnerabilities emulates by the honeypots.


\section{Weaknesses and Strengths of Honeypots}

\section{Honeypots, Elements in Hybrid Machine Learning S.I.E.M}

\section{Conclusions and Future Works}

\begin{thebibliography}{2}
%
\bibitem{dionaea}
Bruce, K.B., Cardelli, L., Pierce, B.C.:
Comparing Object Encodings.
In: Abadi,~M., Ito,~T. (eds.):
Theoretical Aspects of Computer Software.
Lecture Notes in Computer Science, Vol.~1281.
Springer-Verlag, Berlin Heidelberg New York (1997) 415--438
%
\bibitem{kippo}
Bruce, K.B., Cardelli, L., Pierce, B.C.:
Comparing Object Encodings.
In: Abadi,~M., Ito,~T. (eds.):
Theoretical Aspects of Computer Software.
Lecture Notes in Computer Science, Vol.~1281.
Springer-Verlag, Berlin Heidelberg New York (1997) 415--438
%
\end{thebibliography}

\end{document}

